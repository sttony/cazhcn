\chapter{传递闭包、任意两点的最短路径}\label{Sec:Chapter9}
\section{概述}
本章学习两个相关的问题,这两个问题可以非正式的描述为下面两个关于
一个图的所有\emph{顶点对}的问题的答案:
\begin{enumerate}
\item u到v有路径吗?
\item u到v的\emph{最短}路径?
\end{enumerate}

在第七章和第八章我们见过了这两个问题的特殊情况,即第一个顶点是指定
的,仅第二个顶点可以是图中任意顶点。本章中,我们学习更全面的问题。

本章表现的主要算法思想有非常广泛的应用。Klneene(为了综合正则
语言,本书没有涉及),Warshall(为了传递闭包),Floyd(为了所有对
的最短路径)各自独立的从不同的应用发现本章的算法思想。通常,被称为
Kleene-Floyd-Warshall算法。整个这类问题都可以称为
\emph{semi-ring closure}问题。这超出了本书的范围。阅读本章后面的
注释和引用文献得到进一步信息

\section{二元关系的传递闭包}
本节中,我们定义术语二元关系的传递闭包,并将他们的关系看作有向图
的路径。我们也引入一些本章使用的符号。首先检查一种非常直接的算
传递闭包的方法,之后会有复杂的方法。

\subsection{定义和背景}
令S是元素$s_1$、$s_2$、...的集合。复习\ref{Sec:SetTupleRelation}小节,S上
的二元关系是一个S×S子集,称为A。如果 ,我们称si与sj有关系A,使用符号siAsj表示。

假设S有n个元素。关系A可以表示$n\times n$布尔矩阵
\begin{displaymath}
a_{ij}=\left\{\begin{array}{l}
                   true \qquad\mbox{如果 } s_iAs_j\\
                   \raggedleft false \qquad\mbox{其他}
                 \end{array}\right.
\end{displaymath}
开始我们用这种表示方式,后来我们也考虑使用1,0代替\emph{true}和\emph{false};
图表中也显示为1,0。对于布尔矩阵,术语\emph{0矩阵},表示为0,意味着矩阵所有
的元素都是\emph{false};\emph{单位矩阵},表示为I,意味着矩阵对角线的
元素是true,其他的元素都是false。

图中顶点集合之间的邻接关系
\section{传递闭包的Warshall's 算法}
\section{图的任意两点的最短路径}\label{Sec:AllParisShortestPathsinGraphs}
在第8章中,我们学习了Dijkstra's算法(算法8.2),其在带权图中找到了特定
源顶点到所有其他顶点的一条最短路径和距离。算法使用邻接表结构,在
最坏情况下运行在 时间。现在我们考虑下面的问题:

\begin{problem}
任意两顶点之间的最短路径

给出带权图G=(V, E, W),有顶点$V={v_1, \cdots, v_n}$,图用如下权矩阵表示
\begin{equation}
w_{ij})=\left\{\begin{array}{ll}
                   W(v_iv_j)           &v_iv_j \in E\\
                   \infty              &v_iv_j \not\in E \wedge  i\neq j\\
                   \min(0,W(v_iv_j))   &v_iv_j \in E \\
                   0                   &v_iv_j \not\in E
                 \end{array}
        \right.
\end{equation}

计算$n\times n$矩阵D,其中$d_{ij}$表示$v_i$到$v_j$的最短距离。(\emph{距离}就是
最小权路径的权)。
\end{problem}

图9.3是个例子。问题可以扩展成需要最短路径的路由表。如果存在负权的环,则
有些顶点之间的距离不会是最短的路径--在环上绕任意圈得到的路径会更小。

计算D的(如果G没有负权)一个想法是对每一个顶点使用算法8.2。但是,我们可以
使用扩展的Warshall's算法,就是R.W.Floyed。这个算法更精简,它消除了算法8.2的
数据结构。


我们如何计算D[i][j]?最短路径可能以任意顺序经过任意其他顶点。就像在
Warshall's算法中的,我们以路径中间的数字最高的顶点来分类路径(参看定义9.2)。

回顾引理8.5中的最短路径属性:如果vi到vj的最短路径经过中间顶点vk,则路径中
$v_i$到$v_k$中的小段和$v_k$到$v_j$中的小段分别都是最短路径。如果我们选择k作为
$v_i$到$v_j$路径中最高索引的中间顶点(假设路径包含一条以上的边),则出现的
每一个小段中最高数字的中间顶点的索引必然都小于k。(参看图9.2,其展示了
Warshall's算法一样的思想。)这提示在循环中计算矩阵D,以下面的递归等式。
\begin{equation}
\begin{aligned}
&D^{(0)}[i][j]=w_{ij} \\
&D^{(k)}[i][j]\min(D^{(k-1)}[i][j], D^{(k)}[i][j]+D^{(k-1)}[k][j])
\end{aligned}
\end{equation}

这里$w_{ij}$在等式(9.2)中定义。通过前面对最短路径属性的观察,同样的
观点用在了引理9.1中,下面的引理可以证明(证明留在练习中)。

\begin{lemma}
对于每一个$0,\cdots,n$中的k,令$d_{ij}^{(k)}$是$v_i$到$v_j$经过中间顶点$v_k$
的简单路径的最小权,令$D^{(k)}[i][j]$由等式(9.3)定义。则$D^{(k)}[i][j]\leq d_{ij}^{(k)}$ 。
\end{lemma}

\begin{example}
 计算距离矩阵
\end{example}

等式(9.3)计算了一个矩阵序列:$D^{(0)},D^{(1)},\cdots, D^{(n)}$。既然计算
$D^{(k)}$仅用到了$D^{(k-1)}$,我们没必要存储前面的矩阵,只需要两个$n\times n$
矩阵就可以了。事实上,仅要一个;计算可以在矩阵D内进行。既然矩阵的元素仅能不断
变小,如果假设$D^{(k-1)}[i][j]$被使用了,但是$D^{(k)}[i][j]$可以代替它的位置,
我们令$D^{(k)}[i][j] \leq D^{(k-1)}[i][j] \leq d_{ik}^{(k-1)}$,计算过程会
找到更好的路径的。

\begin{algorithm}
(Floyed)任意两点间最短路径
\end{algorithm}

\section{通过矩阵操作计算传递闭包}
